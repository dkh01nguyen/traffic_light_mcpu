\section{Conclusion}
Through the completion of five distinct tasks by implementing a omprehensive toolset for analyzing Petri nets, the team has demonstrated
the following key achievements:

\begin{enumerate}[label = -]
    \item We successfully implemented parsers for PNML files and handled matrix operations, ensuring that the system can 
        process various Petri net structures, including those with edge cases like empty inputs or mismatched dimensions.
    \item By implementing both Explicit (BFS/DFS) and Symbolic (BDD) reachability algorithms, we provided concrete evidence 
        supporting the theoretical trade-offs between execution speed and memory scalability. While Explicit methods proved faster for small-scale 
        nets, the BDD implementation laid the groundwork for handling complex state-space explosions in larger systems.
    \item The successful deployment of Task 4 (Deadlock Detection) and Task 5 (Optimization) showcased the power of hybrid approaches. 
        By combining sequential simulation with symbolic verification, we achieved a deadlock detection system that is both accurate and resistant to infinite loops. 
\end{enumerate}

In summary, this project reinforced our understanding of discrete event systems and formal verification but also highlighted the practical importance of choosing 
the right data structures — Explicit vectors for speed in small systems versus Binary Decision Diagrams for scalability in complex environments. The developed Python application 
stands as a functional prototype for analyzing the behavior, safety, and optimal configurations of concurrent systems modeled by Petri nets.